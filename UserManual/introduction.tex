The Geospatial Data Processor and Visualizer is, as the name suggests, a geopspatial data processor and visualizer. It is a service used to visualize disaster and weather data in real-time. The weather center shows 3-hourly temperature, humidity, wind and precipitation data, while the disaster center shows earthquakes, floods, drought and fire data.\\

The Geospatial Data Processor and Visualizer can help you in three ways: 

\begin{itemize}
	\item It can show you active disasters, which you may track and alert people of impending danger.
	\item It alerts the user of any newly active disasters which you can again track. 
	\item It shows weekly weather forecasts which you may analyse to plan for your week.
\end{itemize}
The Geopsatial Data Processor and Visualizer is developed under Linux, but is set-up to be highly portable using Docker container. As a result, it runs on most operating systems that have support for Docker.\\
This manual is divided into two parts, each of which is divided into several sections.\\ \\
\emph{The first part forms a user manual:}\\ \\
This part will focus on the functionality of the website from the general users perspective.
\begin{itemize}%[listparindent=1.5em, labelsep=2em, itemindent=1.5em]
	\item \textbf{Page Navigation Section}\\ shows you how to navigate around the website.
	\item \textbf{Features}\\ will explain the features of the website in detail. This includes logging in, registering and making queries etc.
\end{itemize}
\emph{The second part provides information for developers:}\\ \\
This part will focus on setting up the servers and overall configuration from the developers perspective. It also supplies information on testing and deployment.
\begin{itemize}%[listparindent=1.5em, labelsep=2em, itemindent=1.5em]
	\item \textbf{Installation Section}\\ discusses how to download, compile, install and configure the service on your platform.
	\item \textbf{Code Maintenance Section}\\ describes the file structure and how you can maintain the code to add features.
	\item \textbf{Troubleshooting Section}\\ discusses how to fix issues that may arise while setting up.
\end{itemize}