\documentclass[a4paper,12pt]{article}
\usepackage{graphicx}
\graphicspath{ {../images/UserManual/} }
\usepackage{hyperref}
\hypersetup{
    colorlinks=true,
    linkcolor=blue,
    filecolor=magenta,      
    urlcolor=blue,
}
 
\urlstyle{same}

\newcommand{\tab}[1]{\hspace{.1\textwidth}\rlap{#1}}

\begin{document}
	
\begin{titlepage}
	\newcommand{\HRule}{\rule{\linewidth}{0.5mm}} % Defines a new command for the horizontal lines, change thickness here

	\center % Center everything on the page
	 
	
	%----------------------------------------------------------------------------------------
	%	TITLE SECTION
	%----------------------------------------------------------------------------------------

	
	{ \huge \bfseries User Manual}\\\HRule \\[0.4cm] % Title of your document
	\Large \textbf{Real-time Geospatial Data Processor and Visualiser} \\
	\small \emph{\textbf{Client: Werner Raath}}
	\HRule \\[1.5cm]
	 
	%----------------------------------------------------------------------------------------
	%	MEMBERS, TEAM NAME SECTION
	%----------------------------------------------------------------------------------------
	\includegraphics[width=\textwidth]{../name} \\[1cm]
	\begin{minipage}{0.4\textwidth}
	\begin{flushleft} \large
	\includegraphics[width=\textwidth]{../logo} \\[0.5cm]
	{\large 23 October 2016}\\
	{\large v1.0}
	\end{flushleft}
	\end{minipage}
	~
	\begin{minipage}{0.5\textwidth}
	\begin{flushright} \large
	\emph{Members:}\\% add your name
	Nsovo Baloyi 12163262

	Maluleki Nyuswa 13040686
	
	Keletso Molefe 14222583
	
	Kamogelo Tswene 12163555

	\end{flushright}
	\end{minipage}\\[4cm]
\end{titlepage}


	\newpage
	
	%-------------------------------------------------------------------------------------
	%		TABLE OF CONTENTS
	%-------------------------------------------------------------------------------------
	\tableofcontents
	\newpage
	\section*{Document History}
	\addcontentsline{toc}{section}{\protect\numberline{}Document History}
	
	\begin{table}[h!]
		
		\centering % used for centering table
		\begin{tabular}{c c c c} % centered columns (4 columns)
			\hline\hline %inserts double horizontal lines
			Version & Date & Changed By & Summary \\ [0.5ex] % inserts table
			%heading
			\hline % inserts single horizontal line
			v0.1 & 29 July 2016 & Nsovo Baloyi & First Draft 
			\\ & & Maluleki Nyuswa &  
			\\ & & Keletso Molefe &
			\\ & & Kamogelo Tswene & \\ [1ex] 
			\hline
			v0.2 & 09 September 2016 & Nsovo Baloyi & Second Draft 
			\\ & & Maluleki Nyuswa &  
			\\ & & Keletso Molefe &
			\\ & & Kamogelo Tswene & \\ [1ex] 
			\hline
			v1.0 & 23 September 2016 & Nsovo Baloyi & Final Draft 
			\\ & & Maluleki Nyuswa &  
			\\ & & Keletso Molefe &
			\\ & & Kamogelo Tswene & \\ [1ex] 
			\hline \hline
		\end{tabular}
		\label{table:nonlin} % is used to refer this table in the text
	\end{table}

	\newpage
	
	%-------------------------------------------------------------------------------------
	%		INTRODUCTION
	%-------------------------------------------------------------------------------------
	\section{Introduction}
	This document contains the user manual for the Geospatial Data Processor and Visualizer System that was developed over a period of 4-5 months for the Software Engineering (COS301) module year project at the University of Pretoria 2016, by the group Coeus.
	
	%-------------------------------------------------------------------------------------
	%		VISION
	%-------------------------------------------------------------------------------------
	\section{Vision}
	The aim of this project is to provide a fully functional system to facilitate the processing and visualisation of geospatial data in real-time, including but not limited to climate and disaster data. The system should be maintainable, with detailed supporting documentation for the Geospatial Data Processor and Visualiser system. The system should be able to store, process and visualise large amounts of data in real-time.
	
	%-------------------------------------------------------------------------------------
	%		BACKGROUND
	%-------------------------------------------------------------------------------------
	\section{Background}
	In the business of disaster management, being able to monitor a disaster's spread and change in near-real time is essential for assessing damages e.g. wildfire burned area, flooded area, etc. and risk to human lives and assets.\\
	There are numerous mature open-source technology for processing the voluminous real-time data, but systems to visualise this data is woefully incomplete and its development remains at an infancy stage. Thus we need to contribute to this quest for knowledge. The need for systems such as the one suggested is very high. The system can save lives and serve as a tool to do research amongst other things.
	
	%-------------------------------------------------------------------------------------
	%		SYSTEM OVERVIEW
	%-------------------------------------------------------------------------------------
	\section{System Overview}		
	This purpose of this document is to provide a detailed summary of the architectural requirements for the Real-time Geospatial Data Processor and Visualiser system in a technology neutral design specification. The specification shows how the system components will communicate with each other and through what means.
		
	%-------------------------------------------------------------------------------------
	%		PART1
	%-------------------------------------------------------------------------------------
	\section{Part1: Using the website }
	\subsection{Website Navigation}
When you first land on the site you will land on the page shown below. The purpose of this section is to describe what each of the labelled functions are used for. \\[0.5cm]
\includegraphics[width=\textwidth]{landingPage} \\[0.5cm]

\begin{enumerate}
	\item \textbf{Menu} 
	\item \textbf{Zoom Button} : With the zoom button feature, one may zoom, the map area, in and out. The "-" button is used for the function of zooming the map area out, and the "+" button is used for the function of zooming the map area in.
	\item \textbf{Map Area} : Display the map
	\item \textbf{Search Area} : Used to search for any location in the world
	\item \textbf{User menu} : The user menu is used to log the user in and out as well as check the settings. 
	\item \textbf{Center Navigation} : The Dropdown area is used to choose which functionality, of the site, you'd like to use. When loaded, the site lands on the weather center by default.
\end{enumerate}

\subsection{Features}
\subsubsection{Login}
\subsubsection{Registration}
\subsubsection{Location Search}
\subsubsection{Disaster Center}
\subsubsection{Weather Center}
\subsubsection{Map Features}
		
	%-------------------------------------------------------------------------------------
	%		PART2
	%-------------------------------------------------------------------------------------
	
	\section{Part2: Setting up}
	\subsection{Installation}
This section will cover the configuration of the various servers on your platform. 
\subsubsection{Minimum Software Requirements}
To run the service you are required to have \href{https://git-scm.com}{git}, \href{https://nodejs.org}{NodeJS} v6 or higher (Which comes with \href{https://www.npmjs.com}{npm} v3 or higher), \href{https://www.rabbitmq.com/download.html}{RabbitMQ}, \href{https://www.mongodb.com/download-center?jmp=nav#community}{MongoDB} and \href{http://rethinkdb.com/docs/install}{RethinkDB} installed on your platform.
\subsubsection{Configuration}
	\begin{enumerate}
		\item \textbf{Download Source Code} \\
		The source code is located on \href{https://github.com}{Github} at the : \url{https://github.com/Coeus2016/geospatial-data-processor-and-visualizer}. You are required to clone the repository. There are two ways to clone the repository:\\
		\begin{itemize}
			\item By using the git command line. In the root folder run the following commands
			\begin{verbatim}
			$git clone https://github.com/Coeus2016/geospatial-data-processor-and
			-visualizer.git		
			\end{verbatim}
		
		\item By downloading the a *zip file from this url \url{https://github.com/Coeus2016/geospatial-data-processor-and-visualizer}\\
		\includegraphics[width=\textwidth]{clone} \\[0.5cm]
		
		\end{itemize}
		
		\item \textbf{Installing Required Modules}\\
		Now that you have downloaded the repository, navigate to the root folder and perform the following commands.
		\begin{verbatim}
			$git submodule update --init --recursive
			$npm install			
			\end{verbatim}
			These commands updated the submodules and install the node modules required to run the server	
	\end{enumerate}

\subsubsection{Starting The Service}
Once you have downloaded and installed all required modules, you are ready to start the service. The following steps need to be followed to insure a safe start of the system.
\begin{enumerate}
	\item Before you can start the application server, you need to make sure all other used services (RethinkDB, MongoDB, RabbitMQ) have been started\footnote{A lot of the time they are configured to start when the computer is booted otherwise you are required to start the services manually in different terminal windows.}. 
	\item Once the other services have been started, running the following command will start the application. 
	\begin{verbatim}
		$npm start			
	\end{verbatim}
	The backend Server will run on port 3200, the scrapper on port 3500 and the webserver on port 3000. To view the website, simply visit \url{http://localhost:3000} on your browser.
\end{enumerate}
\subsection{Code Maintenance}
\subsubsection{Overview of File Structure}
\includegraphics[width=\textwidth]{overall_file_structure} \\[0.5cm]
\begin{itemize}
	\item \textbf{visualizer-server} \\
	The back-end server, visualizer-server is git sub-module. This means that it is located in a different repository location entirely \href{https://github.com/Coeus2016/visualizer-server}{source}. After cloning the repository the developer may edit and add to the code as they wish. It runs a NodeJS server using ExpressJS 
	
	\item \textbf{web-interface}
	The front-end, visualizer-web-interface like the server is also a git sub-module. This means that it is located in a different repository location entirely \href{https://github.com/Coeus2016/visualizer-web-interface}{source}. After cloning the repository the developer may edit and add to the code as they wish. The current framework being used is AngularJS
\end{itemize}

\subsection{Troubleshooting}
\subsubsection{Installation}
While installing the node modules you may run into a window such as this:
\includegraphics[width=\textwidth]{error_install} \\[0.5cm]
This is a common issue and requires you to run the installation script again
\begin{verbatim}
	$npm install
\end{verbatim}


	
	
	\end{document}
