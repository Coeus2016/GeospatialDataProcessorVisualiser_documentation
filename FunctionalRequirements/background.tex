In the business of disaster management, being able to monitor a disaster's spread and change in near-real time is essential for assessing damages e.g. wildfire burned area, flooded area etc. and risk to human lives and assets.\\
There are numerous mature open-source technology for processing the voluminous real-time data, but systems to visualise this data is woefully incomplete and its development remains at an infancy stage. Thus we need to contribute to this quest for knowledge. The need for systems such as the one suggested is very high. The system can save lives and serve as a tool to do research amongst other things.