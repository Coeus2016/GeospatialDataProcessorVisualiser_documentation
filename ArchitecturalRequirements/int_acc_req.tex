
This section of the document outlines the operability of the system running on the server and demonstrates user access and the integration of different technologies. A Web interface will be the single access channel provided to the user. A detailed discussion follows.\\
\paragraph{Web Interface}
The web interface will be accessed typically through a personal computer using a browser such as Google Chrome, Mozilla Firefox and Internet explorer to name a few.\\
The system will be accessed in the following manner:
\begin{enumerate}
	\item The user will open a web browser of their choice.
	\item Click on the web page where the system will be hosted.
	\item A login page will be displayed.
	\item After user information validation, the user can then have full access to the information they wish to access, based on the privileges assigned to them during user creation.
\end{enumerate}
\subsubsection{Integration Channel Requirements}
Two servers would be used to achieve data persistence, one for interfacing with the website and a second for interfacing with the third-party API for data downloading. The system will also have to interface with a map database (Open Street Map) for downloading maps that will be overlaid with geospatial data. \\
\subsubsection{Protocols}
The only protocol that will be used is HTTP for both the web interface and Android interface. 


The figure below is a simplified explanation of the above text.
\begin{center}
	\includegraphics[scale=0.5]{arch/access_channel.png}
\end{center}