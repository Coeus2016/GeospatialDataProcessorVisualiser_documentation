The architectural pattern that will be used for the Real-time Geospatial Data Processor and Visualiser system will be the Layering architectural pattern. \newline
For the Layering pattern, the layers that will be used are
\begin{enumerate}
	\item Client
		\begin{enumerate}
			\item This layer will use the lower layers to have data delivered to it so that it can display the different functionality the user requires, which could range from queries to just simply interacting with a map to view what is going on around it.
		\end{enumerate}
	\item Access
		\begin{enumerate}
			\item This layer will provide services to the Client layer. To do that ExpressJ, which is a NodeJS web application server framework, will be used.
		\end{enumerate}
	\item Business Processes
		\begin{enumerate}
			\item The Business Processes layer will help process requests received from the Client layer and will also contain an API server which will be used to extract data from weather APIs.
		\end{enumerate}
	\item Persistence Access
		\begin{enumerate}
			\item This layer will use Thinky to write data to the Rethink database and RabbitMQ to queue messages to the business process layer.
		\end{enumertae}
	\item Persistence
		\begin{enumerate}
			\item The persistence layer will manage persistent real-time data which will have been acquired from public weather APIs. It would in turn supply data, via the Persistence Access Layer, to the Business Logic layer upon request and when there are updates to the requested data.
		\end{enumerate}
\end{enumerate}


The Layering pattern is used because:
\begin{enumerate}	
	\item it allows applications to be decomposed into groups of subtasks, each group of subtasks at a certain level of abstraction,
	\item the layers are pluggable and replaceable,
	\item complexity is reduced,
	\item the is loose high-level coupling,
	\item the is ability to mock out lower level layers, and
	\item the is enhanced maintainability
\end{enumerate}