The architectural pattern that will be used for the Real-time Geospatial Data Processor and Visualiser system will be the Layering architectural pattern. (See Figure 1)\newline
For the Layering pattern, the layers that will be used are
\begin{enumerate}
	\item Client
		\begin{itemize}
			\item This layer will use the lower layers to have data delivered to it so that it can display the different functionality the user requires in a user friendy manner, which could range from queries to just simply interacting with a map to view what is going on around it.
		\end{itemize}
	\item Access
		\begin{itemize}
			\item This layer, which is a wrapping layer of the business processes layer, makes services available in a technology-neutral way over the Internet. ExpressJS, which is a NodeJS web application server framework, will be used.
		\end{itemize}
	\item Business Processes
		\begin{itemize}
			\item The Business Processes layer will help process requests and will also contain an API server which will be used to extract data from Third Party APIs and/or other reliable data sources.
		\end{itemize}
	\item Persistence Access 
		\begin{itemize}
			\item This layer will be used to access data from the persistence layer, using RabbitMQ to queue messages from the database and it will also be used to write data pulled from public web APIs to the database using Thinky.io.
		\end{itemize}
	\item Persistence
		\begin{itemize}
			\item The persistence layer will manage persistent real-time data which will have been acquired from Third Party APIs. It would in turn supply data, via the Persistence Access Layer, to the Business Logic layer upon request and when there are updates to the requested data.
		\end{itemize}
\end{enumerate}


The Layering pattern is used because:
\begin{enumerate}	
	\item it allows applications to be decomposed into groups of subtasks, each group of subtasks at a certain level of abstraction,
	\item the layers are pluggable and replaceable,
	\item complexity is reduced,
	\item there is loose high-level coupling,
	\item there is ability to mock out lower level layers, and
	\item there is enhanced maintainability
\end{enumerate}